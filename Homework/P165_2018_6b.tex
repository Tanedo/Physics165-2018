\documentclass[12pt]{article}
%% arXiv paper template by Flip Tanedo
%% last updated: Dec 2016



%%%%%%%%%%%%%%%%%%%%%%%%%%%%%
%%%  THE USUAL PACKAGES  %%%%
%%%%%%%%%%%%%%%%%%%%%%%%%%%%%

\usepackage{amsmath}
\usepackage{amssymb}
\usepackage{amsfonts}
\usepackage{graphicx}
\usepackage{xcolor}
\usepackage{nopageno}
\usepackage{enumerate}
\usepackage{parskip}
\usepackage{framed}
%\usepackage{bbm} 
\usepackage[normalem]{ulem}


\renewcommand{\thesection}{}
\renewcommand{\thesubsection}{\arabic{subsection}}

%%%%%%%%%%%%%%%%%%%%%%%%%%%%%%%%%%%%%%%%%%%%%%%
%%%  PAGE FORMATTING and (RE)NEW COMMANDS  %%%%
%%%%%%%%%%%%%%%%%%%%%%%%%%%%%%%%%%%%%%%%%%%%%%%

\usepackage[margin=2cm]{geometry}   % reasonable margins

\graphicspath{{figures/}}	        % set directory for figures

% for capitalized things
\newcommand{\acro}[1]{\textsc{\MakeLowercase{#1}}}    

%\numberwithin{equation}{section}    % set equation numbering
\renewcommand{\tilde}{\widetilde}   % tilde over characters
%\renewcommand{\vec}[1]{\mathbf{#1}} % vectors are boldface

\newcommand{\dbar}{d\mkern-6mu\mathchar'26}    % for d/2pi
\newcommand{\ket}[1]{\left|#1\right\rangle}    % <#1|
\newcommand{\bra}[1]{\left\langle#1\right|}    % |#1>
\newcommand{\Xmark}{\text{\sffamily X}}        % cross out

\let\olditemize\itemize
\renewcommand{\itemize}{
  \olditemize
  \setlength{\itemsep}{1pt}
  \setlength{\parskip}{0pt}
  \setlength{\parsep}{0pt}
}


% Commands for temporary comments
\newcommand{\comment}[2]{\textcolor{red}{[\textbf{#1} #2]}}
\newcommand{\flip}[1]{{\color{red} [\textbf{Flip}: {#1}]}}
\newcommand{\email}[1]{\texttt{\href{mailto:#1}{#1}}}

\newenvironment{institutions}[1][2em]{\begin{list}{}{\setlength\leftmargin{#1}\setlength\rightmargin{#1}}\item[]}{\end{list}}


\usepackage{fancyhdr}		% to put preprint number



% Commands for listings package
%\usepackage{listings}      % \begin{lstlisting}, for code
%
% \lstset{basicstyle=\ttfamily\footnotesize,breaklines=true}
%    sets style to small true-type



%%%%%%%%%%%%%%%%%%%
%%%  HYPERREF  %%%%
%%%%%%%%%%%%%%%%%%%

%% This package has to be at the end; can lead to conflicts
\usepackage{microtype}
\usepackage[
	colorlinks=true,
	citecolor=black,
	linkcolor=black,
	urlcolor=green!50!black,
	hypertexnames=false]{hyperref}





\begin{document}


\begin{center}

    {\Large \textsc{Long HW 6}:
    \textbf{Putting it all together}}
    
\end{center}

\vskip .4cm

\noindent
\begin{tabular*}{\textwidth}{rl}
	\textsc{Course:}& Physics 165, \emph{Introduction to Particle Physics} (2018)
	\\
	\textsc{Instructor:}& Prof. Flip Tanedo (\email{flip.tanedo@ucr.edu})
	\\
	\textsc{Due by:}& \textbf{Tuesday}, February 20 
\end{tabular*}

\noindent
This is the main weekly homework set. Unless otherwise stated, give all responses in natural units where $c = \hbar = 1$ and energy is measured in electron volts (usually MeV or GeV). 

Recall that a Lagrangian Density (``Lagrangian'' from now on) is a function $\mathcal L(x)$ that is integrated over spacetime to give the action, $S$:
\begin{align}
	S = \int d^4x \mathcal L(x) \ .
\end{align}
In fact, to be precise, $\mathcal L$ is a \emph{functional} of the \textbf{fields} in your theory, $\mathcal L[\varphi(x), \cdots]$. The quadratic part of the Lagrangian is special because it can be solved exactly to derive the \textbf{propagator} of a particle. 



%\subsection{Spring Theory}
%
%In class we wrote down the Lagrangian for an infinite array of beads connected to their nearest neighbors by springs:
%\begin{align}
%	L = \frac 12 \sum_i \Delta x
%	\left[
%	{\dot q}_i^2 - \left(\frac{q_{i+1}-q_i}{\Delta x}\right)^2 
%	\right] \ ,
%\end{align}
%where we've chosen units to fix the relative prefactor. In the continuum limit, the bracketed turns into $(\partial q)^2$. The beads may have had some mass $m_\text{bead}$, but those got absorbed into our choice of units. 
%
%\subsubsection{Dimensional Analysis}
%
%Assume that the springs live in a three-dimensional array. That is to say, they live in four-dimensional spacetime. 
%
%\subsection{Lagrangian Densities}



\subsection{Why the mass term is a mass}

In class we wrote down the Lagrangian for an infinite array of beads connected to their nearest neighbors by springs:
\begin{align}
	L = \frac 12 \sum_i \Delta x
	\left[
	\frac{m_\text{bead}}{\Delta x}
	{\dot q}_i^2 
	- 
	k\Delta x
	\left(\frac{q_{i+1}-q_i}{\Delta x}\right)^2 
	\right] \ ,
\end{align}
we then `chose units' so that this became
\begin{align}
	L &= \frac 12 \sum_i \Delta x
	\left[
	{\dot q}_i^2 - \left(\frac{q_{i+1}-q_i}{\Delta x}\right)^2 
	\right]  
	\quad\to\quad \int dx \, \frac{1}{2} (\partial q)^2 \ .
\end{align}
We identify $\mathcal L[q] = (\partial q)^2/2$.
In the case of a (3+1)-dimensional spacetime, we this generalizes to
\begin{align}
	S = \int dt \,d^3x\, \mathcal L = \int d^4x \, \frac 12 (\partial q)^2 \ .
\end{align}
We propose adding another term to the Lagrangian,
\begin{align}
	\Delta \mathcal L = - \frac 12 A q^2 \ .
\end{align}

\subsubsection{Dimensional Analysis}
In this class, `dimension' means ``mass dimension.'' For example, an energy $E$ has dimension 1, while a length $\mathcal\ell$ has dimension $-1$. The notation for this is
\begin{align}
	[E] &=1 & [\ell] &=-1 \ .
\end{align} 
This just means that you can write a quantity $x$ as a number times $\text{GeV}^{[x]}$. 

\begin{enumerate}
	\item[(a)] Given that $[S]=0$, what is the dimension of $\mathcal L$?
	\item[(b)] What is the dimension of the field $q$?
	\item[(c)] What is the dimension of the coefficient $A$?
\end{enumerate}

\subsubsection{What is mass?}

Write $A$ as some mass scale $m$ to the appropriate power. We want to identify $m$ with the mass of a quantum excitation. The equation of motion for
\begin{align}
	\mathcal L = \frac 12\left(\partial q\right)^2 - \frac 12 A q^2 \ ,
\end{align}
is simply
\begin{align}
	(-\partial^2 - A) q = 0 \ .
\end{align}
To understand what this means, make the ansatz that $q$ is a plane wave of definite momentum, $q(x)\propto \exp(ip^\mu x_\mu)$. Show that the equation of motion then implies that $m$ is identified with the `mass' of the plane wave. \textsc{Hint}: use the Einstein relation for an on-shell particle.



\subsection{Mixed Mass}

Consider the following quadratic Lagrangian written in terms of fields $\varphi_1(x)$ and $\varphi_2(x)$.
\begin{align}
	\mathcal L[\varphi_1, \varphi_2] &=
	\frac{1}{2}\left(\partial_\mu\varphi_1\right)
	\left(\partial^\mu\varphi_2\right)
	+
	\frac{1}{2}\left(\partial_\mu\varphi_2\right)
	\left(\partial^\mu\varphi_2\right)
	- \frac 12 m^2\varphi_1\varphi_2 \ .
\end{align}
Observe that the mass term connects a $\varphi_1$ and a $\varphi_2$ field. This means that these field mix with one another. We are free to redefine fields. Show that the following redefinition diagonalizes the quadratic Lagrangian above; that is: each term contains only one type of field.
\begin{align}
	\varphi_1(x) &= \frac{1}{\sqrt{2}}\left(\varphi_A(x) + \varphi_B(x) \right)
	\\
	\varphi_2(x) &= \frac{1}{\sqrt{2}}\left(\varphi_A(x) - \varphi_B(x) \right)
\end{align}
What is the mass-squared of the $\varphi_A$ field? What is strange about the mass-squared of the $\varphi_B$ field?

\subsection{Negative mass-squared}

A field that has a negative value for $m^2$ is problematic. That would imply that the mass is imaginary. Not good. We need to make sense of this. Assume you have the following Lagrangian:
\begin{align}
	\mathcal L[\varphi] &=
	\frac{1}{2}\left( \partial_\mu\varphi\right)\left(\partial^\mu \varphi\right) - V[\varphi]
	&
	V[\varphi] &=
	-\frac{1}{2} m^2 \varphi^2 + \frac{\lambda}{4} \varphi^4 \ .
\end{align}
Here $V[\varphi]$ is called the \textbf{potential} of the field $\varphi$. It gives the potential energy of a [classical] field configuration.
\begin{enumerate}
	\item[(a)] Plot the potential $V[\varphi]$ as a function of $\varphi$.
	\item[(b)] For what value(s) of $\varphi$ is $V[\varphi]$ minimized? Call these values $\pm\varphi_0$. These are just some spacetime-independent constants.
	\item[(c)] The problem with the field $\varphi(x)$ is that excitations of the field are not being expanded about the minimum energy (potential) configuration. Instead, define a shifted field $\phi(x)$ by $\varphi(x) = \varphi_0 + \phi(x)$. Expand $V[\varphi_0 + \phi(x)]$, show that the $\phi(x)$ field has a positive mass-squared term. What is the mass of the $\phi$ particle?
\end{enumerate}
The lesson here is that in the absence of excitations, the field wants to take values $\varphi(x) = \varphi_0$. This is the constant value that minimized the potential. Quantum excitations of the field are `wiggles' on top of this background value.

The background value, $\varphi_0$ is called the \textbf{vacuum expectation value} of $\varphi(x)$ and is often written $\langle \varphi \rangle = \varphi_0$. An analog of this is precisely whats happening with the Higgs field. 



\subsection{Reading}

Read the article ``Particle Physics and the Standard Model'' by Raby, Slansky, and West in \emph{Particle Physics: A Los Alamos Primer}\footnote{\url{https://archive.org/details/ParticlePhysicsAndTheStandardModel}}. This should be a nice, enjoyable article that contextualizes our discussions so far. Answer the following question: Let $g$ be the coupling of the weak bosons ($W^A$) and $g'$ be the coupling of the hypercharge boson ($B$). If I doubled $g$, what happens to the size of the photon coupling, $e$? What happens to the mass of the $W$ boson? You do not have to read in thorough detail, but we will come back to this article soon.



\appendix
\vspace{1em}
{\Large\textbf{Extra Credit}}






\subsection{Dimensional analysis: propagator to long-range force}

The propagator of an intermediate, massless particle with momentum $p$ is
\begin{align}
	\Delta(p) = \frac{1}{p^2 - m^2} \ .
\end{align}
(We're dropping factors of $\pm i$ that are not our concern for this course.) Draw the Feynman diagrams for an electron with momentum $p_1$ and a positron with momentum $p_2$ scattering into an electron with momentum $k_1$ and a positron with momentum $k_2$,
\begin{align}
	e^-(p_1) + e^+(p_2) \to e^-(k_1) + e^+(k_2) \ .
\end{align} 
Write down the propagator for the internal line for each of these diagrams. 
\begin{enumerate}
	\item[(a)] Can either of the denominators ever be zero?
	\item[(b)] One of these diagrams represents the \emph{long range} electric force of the electron and positron acting on one another. Based on the interpretation of the Feynman diagram as a spacetime trajectory, which diagram represents the long range force?
	\item[(c)] Let $q = p_1 - k_1$ be the exchanged 4-momentum between the electron and positron. The long range potential $V(r)$ between the two particles is given by the spatial Fourier transform of the amplitude $V(\mathbf{r})\sim\int d^3\mathbf{q} \exp\left(\textbf{q}\cdot \textbf{r}\right) \mathcal M$. Assuming that $\mathcal M$ is simply the propagator of the diagram encoding the long-range force, what is the $r$ dependence of the potential? Don't calculate, just use dimensional analysis. 
\end{enumerate}


\subsection{Reading}

Read the article ``Lecture Notes: from simple field theories to the standard model'' by Slansky in \emph{Particle Physics: A Los Alamos Primer}\footnote{\url{library.lanl.gov/cgi-bin/getfile?11-03.pdf}}. Based on the discussion in lecture 8, does the electron have a bigger Yukawa coupling or a smaller Yukawa coupling than the muon? (The Yukawa is what Slansky calls $G_Y$.) You do not have to read in thorough detail, but we will come back to this article soon.

%\appendix
%\vspace{1em}
%{\Large\textbf{Extra Credit}}
%
%\subsection{Error analysis and dimensional analysis}
%
%Consider the `high school' level problem of finding the time it takes for an object to reach the ground after being dropped from rest at height $h$. Assume everything is human scale and on the surface of the Earth so that the zeroth order solution is $t_0 = \sqrt{2h/g}$. Estimate the size of the correction from special relativity. If you need a hint, this problem came from ``Dimensional analysis, falling bodies, and the fine art of not solving differential equations'' by Craig Bohren\footnote{\emph{Am. J. Phys.} \textbf{72}  4 , April 2004 \url{http://dx.doi.org/10.1119/1.1574042}}. 

%\textsc{Hint}: 



\end{document}