\documentclass[12pt]{article}
%% arXiv paper template by Flip Tanedo
%% last updated: Dec 2016



%%%%%%%%%%%%%%%%%%%%%%%%%%%%%
%%%  THE USUAL PACKAGES  %%%%
%%%%%%%%%%%%%%%%%%%%%%%%%%%%%

\usepackage{amsmath}
\usepackage{amssymb}
\usepackage{amsfonts}
\usepackage{graphicx}
\usepackage{xcolor}
\usepackage{nopageno}
\usepackage{enumerate}
\usepackage{parskip}


\renewcommand{\thesection}{}
\renewcommand{\thesubsection}{\arabic{subsection}}

%%%%%%%%%%%%%%%%%%%%%%%%%%%%%%%%%%%%%%%%%%%%%%%
%%%  PAGE FORMATTING and (RE)NEW COMMANDS  %%%%
%%%%%%%%%%%%%%%%%%%%%%%%%%%%%%%%%%%%%%%%%%%%%%%

\usepackage[margin=2cm]{geometry}   % reasonable margins

\graphicspath{{figures/}}	        % set directory for figures

% for capitalized things
\newcommand{\acro}[1]{\textsc{\MakeLowercase{#1}}}    

\numberwithin{equation}{section}    % set equation numbering
\renewcommand{\tilde}{\widetilde}   % tilde over characters
\renewcommand{\vec}[1]{\mathbf{#1}} % vectors are boldface

\newcommand{\dbar}{d\mkern-6mu\mathchar'26}    % for d/2pi
\newcommand{\ket}[1]{\left|#1\right\rangle}    % <#1|
\newcommand{\bra}[1]{\left\langle#1\right|}    % |#1>
\newcommand{\Xmark}{\text{\sffamily X}}        % cross out

\let\olditemize\itemize
\renewcommand{\itemize}{
  \olditemize
  \setlength{\itemsep}{1pt}
  \setlength{\parskip}{0pt}
  \setlength{\parsep}{0pt}
}


% Commands for temporary comments
\newcommand{\comment}[2]{\textcolor{red}{[\textbf{#1} #2]}}
\newcommand{\flip}[1]{{\color{red} [\textbf{Flip}: {#1}]}}
\newcommand{\email}[1]{\texttt{\href{mailto:#1}{#1}}}

\newenvironment{institutions}[1][2em]{\begin{list}{}{\setlength\leftmargin{#1}\setlength\rightmargin{#1}}\item[]}{\end{list}}


\usepackage{fancyhdr}		% to put preprint number



% Commands for listings package
%\usepackage{listings}      % \begin{lstlisting}, for code
%
% \lstset{basicstyle=\ttfamily\footnotesize,breaklines=true}
%    sets style to small true-type



%%%%%%%%%%%%%%%%%%%
%%%  HYPERREF  %%%%
%%%%%%%%%%%%%%%%%%%

%% This package has to be at the end; can lead to conflicts
\usepackage{microtype}
\usepackage[
	colorlinks=true,
	citecolor=black,
	linkcolor=black,
	urlcolor=green!50!black,
	hypertexnames=false]{hyperref}





\begin{document}


\begin{center}

    {\Large \textsc{Short HW 7}:
	\textbf{Fill in the Indices}}
    
\end{center}

\vskip .4cm

\noindent
\begin{tabular*}{\textwidth}{rl}
	\textsc{Course:}& Physics 165, \emph{Introduction to Particle Physics} (2018)
	\\
	\textsc{Instructor:}& Prof. Flip Tanedo (\email{flip.tanedo@ucr.edu})
	\\
	\textsc{Due by:}& \textbf{Thursday}, February 22
\end{tabular*}

\noindent
Note that this short assignment is due in class on Thursday. You have only \emph{two days} to do it. This should be quick, I recommend doing it right after class on Tuesday.


In class we write out the kinetic term for the Higgs boson, $H$. We started with the kinetic term for a complex scalar, $\mathcal L \supset |\partial H|^2 = (\partial_\mu H)^* (\partial^\mu H)$. Then we promoted the derivative to a covariant derivative that `knows' about the gauge symmetries (charges and tensor structure) of the Higgs:
\begin{align*}
	D_\mu = \partial_\mu 
	- i \sum_{\aleph} g_\aleph q_\aleph V^\aleph_\mu  
	- i \sum_{\diamondsuit} \sum_{A=\text{adj.}} g_\diamondsuit W^A_\mu \left(T^A\right)^{\bigtriangleup}_{\phantom{\bigtriangleup}\bigtriangledown}
\end{align*}
Here $\aleph$ runs over all \textbf{Abelian} (charge) gauge symmetries and $\diamondsuit$ runs over all \textbf{non-Abelian} (index) gauge symmetries. $A$ is used as a generic adjoint index, and $\bigtriangleup$/$\bigtriangledown$ are generic indices for a fundamental (column/row vector). All terms that don't have explicit indices are assumed to be proportional to the identity ($\delta^\bigtriangleup_\bigtriangledown$). 

For example, for SU(2) the $T^A$ are
\begin{align*}
	T^1 &= \frac 12
	\begin{pmatrix}
		0  & 1\\
		1 & 0
	\end{pmatrix}
	&
	T^2 &= \frac 12
	\begin{pmatrix}
		0  & -i\\
		i & 0
	\end{pmatrix}
	&
	T^3 &= \frac 12
	\begin{pmatrix}
		1  & 0\\
		0 & -1
	\end{pmatrix} \ .
\end{align*}
The Standard Model Higgs [doublet] field $H$ is:
\begin{itemize}
	\item a doublet (fundamental) under SU(2) weak. Use indices $a$ and $b$ for the fundamental/anti-fundamental indices. The weak force has gauge coupling $g$.
	\item charged $q_Y = 1/2$ under U(1) hypercharge. The hypercharge gauge coupling is $g'$.
\end{itemize}
We conventionally drop all the indices and write the kinetic term of the Higgs boson to be
\begin{align*}
	\mathcal L_\text{kin.}[H] = |DH|^2 \ .
\end{align*}
Write out $DH$ with full indices for the gauge and spacetime/Lorentz symmetries. 


\textbf{Extra credit}: Writing $H=(H_1, H_2)^T$ and using only the kinetic terms, draw the all interaction vertices that include a $W^+$ and an $H_2$. (Confirm that they're all invariant.)

\end{document}