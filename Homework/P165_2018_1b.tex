\documentclass[12pt]{article}
%% arXiv paper template by Flip Tanedo
%% last updated: Dec 2016



%%%%%%%%%%%%%%%%%%%%%%%%%%%%%
%%%  THE USUAL PACKAGES  %%%%
%%%%%%%%%%%%%%%%%%%%%%%%%%%%%

\usepackage{amsmath}
\usepackage{amssymb}
\usepackage{amsfonts}
\usepackage{graphicx}
\usepackage{xcolor}
\usepackage{nopageno}
\usepackage{enumerate}
\usepackage{parskip}


\renewcommand{\thesection}{}
\renewcommand{\thesubsection}{\arabic{subsection}}

%%%%%%%%%%%%%%%%%%%%%%%%%%%%%%%%%%%%%%%%%%%%%%%
%%%  PAGE FORMATTING and (RE)NEW COMMANDS  %%%%
%%%%%%%%%%%%%%%%%%%%%%%%%%%%%%%%%%%%%%%%%%%%%%%

\usepackage[margin=2cm]{geometry}   % reasonable margins

\graphicspath{{figures/}}	        % set directory for figures

% for capitalized things
\newcommand{\acro}[1]{\textsc{\MakeLowercase{#1}}}    

\numberwithin{equation}{section}    % set equation numbering
\renewcommand{\tilde}{\widetilde}   % tilde over characters
\renewcommand{\vec}[1]{\mathbf{#1}} % vectors are boldface

\newcommand{\dbar}{d\mkern-6mu\mathchar'26}    % for d/2pi
\newcommand{\ket}[1]{\left|#1\right\rangle}    % <#1|
\newcommand{\bra}[1]{\left\langle#1\right|}    % |#1>
\newcommand{\Xmark}{\text{\sffamily X}}        % cross out

\let\olditemize\itemize
\renewcommand{\itemize}{
  \olditemize
  \setlength{\itemsep}{1pt}
  \setlength{\parskip}{0pt}
  \setlength{\parsep}{0pt}
}


% Commands for temporary comments
\newcommand{\comment}[2]{\textcolor{red}{[\textbf{#1} #2]}}
\newcommand{\flip}[1]{{\color{red} [\textbf{Flip}: {#1}]}}
\newcommand{\email}[1]{\texttt{\href{mailto:#1}{#1}}}

\newenvironment{institutions}[1][2em]{\begin{list}{}{\setlength\leftmargin{#1}\setlength\rightmargin{#1}}\item[]}{\end{list}}


\usepackage{fancyhdr}		% to put preprint number



% Commands for listings package
%\usepackage{listings}      % \begin{lstlisting}, for code
%
% \lstset{basicstyle=\ttfamily\footnotesize,breaklines=true}
%    sets style to small true-type



%%%%%%%%%%%%%%%%%%%
%%%  HYPERREF  %%%%
%%%%%%%%%%%%%%%%%%%

%% This package has to be at the end; can lead to conflicts
\usepackage{microtype}
\usepackage[
	colorlinks=true,
	citecolor=black,
	linkcolor=black,
	urlcolor=green!50!black,
	hypertexnames=false]{hyperref}





\begin{document}


\begin{center}

    {\Large \textsc{Weekly HW 1}:
    \textbf{Kinematics and QED}}
    
\end{center}

\vskip .4cm

\noindent
\begin{tabular*}{\textwidth}{rl}
	\textsc{Course:}& Physics 165, \emph{Introduction to Particle Physics} (2018)
	\\
	\textsc{Instructor:}& Prof. Flip Tanedo (\email{flip.tanedo@ucr.edu})
	\\
	\textsc{Due by:}& {Tuesday}, January 16
\end{tabular*}

\noindent
This is the main weekly homework set. Unless otherwise stated, give all responses in natural units where $c = \hbar = 1$ and energy is measured in electron volts (usually MeV or GeV). 

\subsection{Everything in natural units}

Write the following quantities in natural units with energy measured in GeV. You may write everything to one significant figure.

\begin{itemize}
	\item The mass of the sun, $M_\odot$.
	\item The present day Hubble expansion rate, $H_0$. 
	\item The classical electron radius, $r_e$. 
	\item The Schwarszchild radius of the sun, $2G_N M_\odot/c^2$.
\end{itemize}

You may look up the information anywhere you want, but I suggest the first few pages of the PDG. 

\subsection{Special Relativity and Kinematics}

\subsubsection{A relativistic electron}

In some frame, the electron has \emph{momentum} equivalent to its rest mass, $m_e$. Use the value of the rest mass to one significant figure. I shouldn't have to tell you where to look it up. Write out the components of the \textbf{momentum four-vector} $p_\mu$. 

\subsubsection{A symmetric particle collider}

Imagine a symmetric electron--positron collider. At the collision point, it collides a beam of electrons and positrons with one another so that these have four-momenta:
\begin{align}
	p_\mu^{e^-} &= (E,0,0,p)
	&
	\text{and}
	&&
	p_\mu^{e^+} &= (E,0,0,-p) \ .
\end{align}
What is the expression for $p$ as a function of $E$ and $m_e$? What is the \textbf{center of mass energy} of the collision in the lab frame?

Suppose that this collider was invented to produce a 91~GeV particle, $Z$, through the process $e^+ e^- \to Z$. What energy $E$ is required for each beam? What is the momentum of the $Z$ particle in the lab frame?

\textbf{Extra credit}: Suppose the $Z$ is unstable and decays. This means that you don't get to measure it directly. Without knowing anything else about how the $Z$ interacts, what is one \textbf{decay mode} that is guaranteed to exist? In other words, what types of particles should you make sure you can detect? 

\subsubsection{A fixed target experiment}


Imagine a very asymmetric kind of collider called a \textbf{fixed target experiment}: a high-energy beam of particles hits a stationary target. Assume that both the beam and the target are composed of protons and that the collision occurs head-on\footnote{This is a classical idea, but for now we can live with this kind of deceit. Relevant: \url{https://www.youtube.com/watch?v=AnaQXJmpwM4}}. Write the four-momenta of a beam particle and the target particle in the lab frame. 

Suppose you wanted to produce some completely made up particle---let's call it a \emph{Flippon}\footnote{Unrelated to this: \url{https://arxiv.org/abs/1602.01377}}---that has a mass of 14 GeV. To one significant figure, what proton beam energy $E$ is required to produce the Flippon through $pp \to \text{Flippon}$? (Assume that such a process is possible.) What is the momentum of the Flippon in the lab frame?

\textsc{Hint}: This problem is constrained by kinematics. There's an easy way and a hard way of doing this. One involves doing a Lorentz transformation to a more convenient frame. The other involves realizing that the quantity in the convenient frame can be written as a Lorentz invariant. I don't care which way you do this, though you should probably to understand how to do it both ways. 

\textsc{Discussion}: Fixed target experiments are nice because you don't have to worry about engineering two beams to collide with one another. They also have a very useful feature that the new particle is produced \emph{boosted} relative to the lab frame. This can be very useful for untangling the decay products of the new particle from other  particle debris from the beam hitting the target.


%\subsubsection{Lorentz Transformations}
%
%What is the relativistic boost factor $\gamma$ of an electron that has been boosted to $E = 1$ GeV? What is the velocity of such an electron? Write out the boost matrix that transforms from the particle's rest frame to the frame where it has $E= 1$ GeV.
%
%\textsc{Hint}: If only there were some reference material in the back of some booklet that had a review of kinematics. 





\subsection{Stability of the electron}

Draw the Feynman diagram for the process $e^- \to e^- \gamma$. Based on this, I foolishly propose that a 1~GeV electron can `decay' into a lower-energy electron and a photon. Prove that this is impossible kinematically. 

\textsc{Hint}: There's an easy way and a hard way of doing this. They both involve conservation of energy and momentum. I don't care which way you do this.


\subsection{Feedback}

Approximately how long did it take you to complete the non-extra credit parts of this assignment?


\section{Extra Credit}

If you do any of these problems, please write a short note giving your thoughts on the reading: did you like them? Were they too simple / difficult? I do not expect you to be able to complete all (or necessarily any) of the extra credit.

\subsection{Minkowski Diagrams}

The mathematical basis of relativity is geometry. This is most simply seen in what are called \textbf{Minkowski diagrams}. I'm pretty sure  A good introduction to these are in \url{https://arxiv.org/abs/1508.01968} by Boxiang Liu and Thushara Perera\footnote{A somewhat more polished reference is the book \emph{Very Special Relativity} by Sander Bais. There's also a Khan Academy video, \url{https://youtu.be/nEqexIckVCM}.}. Consider two reference frames with some non-zero relative velocity. Sketch the axes of the Minkowski diagram this system: that is, draw the $(x,t)$ axes and the $(x',t')$ axis where $(x',t')$ are related to $(x,t)$ by a Lorentz transformation.  Draw two spacetime events and their respective light cones. Comment on the idea of causality using these diagrams. Those who are mathematically inclined may enjoy \url{https://doi.org/10.1119/1.4997027}. 

\subsection{Impact of Special Relativity on Physics: Compton Scattering}

Look over David Jackson's article ``The Impact of Special Relativity on Theoretical Physics'' from the May 1987 issue of \emph{Physics Today}, \url{https://doi.org/10.1063/1.881108}. Focus on the section ``Waves and particles,'' where the author discusses \textbf{Compton scattering}. You drew the Feynman diagrams for this on your short homework assignment \#1. Use special relativity to derive the author's expression for $\delta\lambda$, the shift in the photon wavelength.

\subsection{Relativistic mass}

There is an antiquated notion of \emph{relativistic mass} that people used to talk about. Lev Okun gives a nice overview in ``The Concept of Mass'' in the June 1989 issue of \emph{Physics Today}, \url{https://doi.org/10.1063/1.881171}. Read the article and explain why there is only \emph{one} useful notion of mass and that it is the \emph{rest mass}.


\end{document}