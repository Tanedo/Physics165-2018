\documentclass[12pt]{article}
%% arXiv paper template by Flip Tanedo
%% last updated: Dec 2016



%%%%%%%%%%%%%%%%%%%%%%%%%%%%%
%%%  THE USUAL PACKAGES  %%%%
%%%%%%%%%%%%%%%%%%%%%%%%%%%%%

\usepackage{amsmath}
\usepackage{amssymb}
\usepackage{amsfonts}
\usepackage{graphicx}
\usepackage{xcolor}
\usepackage{nopageno}
\usepackage{enumerate}
\usepackage{parskip}


\renewcommand{\thesection}{}
\renewcommand{\thesubsection}{\arabic{subsection}}

%%%%%%%%%%%%%%%%%%%%%%%%%%%%%%%%%%%%%%%%%%%%%%%
%%%  PAGE FORMATTING and (RE)NEW COMMANDS  %%%%
%%%%%%%%%%%%%%%%%%%%%%%%%%%%%%%%%%%%%%%%%%%%%%%

\usepackage[margin=2cm]{geometry}   % reasonable margins

\graphicspath{{figures/}}	        % set directory for figures

% for capitalized things
\newcommand{\acro}[1]{\textsc{\MakeLowercase{#1}}}    

\numberwithin{equation}{section}    % set equation numbering
\renewcommand{\tilde}{\widetilde}   % tilde over characters
\renewcommand{\vec}[1]{\mathbf{#1}} % vectors are boldface

\newcommand{\dbar}{d\mkern-6mu\mathchar'26}    % for d/2pi
\newcommand{\ket}[1]{\left|#1\right\rangle}    % <#1|
\newcommand{\bra}[1]{\left\langle#1\right|}    % |#1>
\newcommand{\Xmark}{\text{\sffamily X}}        % cross out

\let\olditemize\itemize
\renewcommand{\itemize}{
  \olditemize
  \setlength{\itemsep}{1pt}
  \setlength{\parskip}{0pt}
  \setlength{\parsep}{0pt}
}


% Commands for temporary comments
\newcommand{\comment}[2]{\textcolor{red}{[\textbf{#1} #2]}}
\newcommand{\flip}[1]{{\color{red} [\textbf{Flip}: {#1}]}}
\newcommand{\email}[1]{\texttt{\href{mailto:#1}{#1}}}

\newenvironment{institutions}[1][2em]{\begin{list}{}{\setlength\leftmargin{#1}\setlength\rightmargin{#1}}\item[]}{\end{list}}


\usepackage{fancyhdr}		% to put preprint number



% Commands for listings package
%\usepackage{listings}      % \begin{lstlisting}, for code
%
% \lstset{basicstyle=\ttfamily\footnotesize,breaklines=true}
%    sets style to small true-type



%%%%%%%%%%%%%%%%%%%
%%%  HYPERREF  %%%%
%%%%%%%%%%%%%%%%%%%

%% This package has to be at the end; can lead to conflicts
\usepackage{microtype}
\usepackage[
	colorlinks=true,
	citecolor=black,
	linkcolor=black,
	urlcolor=green!50!black,
	hypertexnames=false]{hyperref}





\begin{document}


\begin{center}

    {\Large \textsc{Short HW 4}:
    \textbf{Leptonic Electroweak Theory}}
    
\end{center}

\vskip .4cm

\noindent
\begin{tabular*}{\textwidth}{rl}
	\textsc{Course:}& Physics 165, \emph{Introduction to Particle Physics} (2018)
	\\
	\textsc{Instructor:}& Prof. Flip Tanedo (\email{flip.tanedo@ucr.edu})
	\\
	\textsc{Due by:}& \textbf{Thursday}, February 1
\end{tabular*}

\noindent
Note that this short assignment is due in class on Thursday. You have only \emph{two days} to do it. This should be quick, I recommend doing it right after class on Tuesday.

\subsection{Rules of the theory}

You may find previous homework assignments useful. In this problem you will motivate the rules of the unbroken leptonic electroweak theory for a single generation\footnote{The words `unbroken' and the reference to `single generation' are hints that we're going to make things more complicated in the near future.}. Your symmetries are as follows:
\begin{enumerate}
	\item Spacetime symmetry: translations in spacetime and Lorentz.
	\item U(1) hypercharge, gauged.
	\item SU(2) weak, gauged.
\end{enumerate}
Your particle content is:
\begin{enumerate}
	\item Hypercharge gauge boson, $B_\mu$, required by the gauged hypercharge symmetry. It is spin-1 with no other charges.
	\item Electroweak gauge boson, $W^A_\mu$, required by the gauged weak symmetry. It is spin-1 with triplet (adjoint) SU(2) weak charge, $A=1,2,3$.
	\item A lepton doublet, $L^{a\alpha}$. This is a spin-1/2 fermion that has hypercharge $Y=-1/2$ and is in the doublet (fundamental) representation of SU(2) weak. The two components have special names, $L^1 = \nu_L$ and $L^2 = e_L$.
	\item A left-handed positron, $\bar E^\alpha$ that has hypercharge $Y=+1$ and carries no SU(2) weak charge. You can think of $\bar E$ as the anti-particle to $(\bar E)^\dag = e_R$, a right-handed electron.
	\item A Higgs doublet, $H^a$, that has hypercharge $Y=+1/2$ and is in the doublet (fundamental) representation of SU(2) weak. For now we can refer to the components as $H^1$ and $H^2$.
\end{enumerate}
Write out all of the allowed three-particle Feynman rules for this theory in terms of $B_\mu$, $W_\mu^A$, $L^{\alpha a}$, $\bar E^\alpha$, and $H^a$. 

\textbf{Extra credit}: write out the allowed three-particle Feynman rules in terms of $B_\mu$, $W_\mu^\pm$, $W^3$, $\nu_L$, $e_L$, $e_R$, and $H^{1,2}$. 

\textbf{Extra credit}: What combination of quantum numbers (charges) in this theory seems to give the correct electric charge `in real life?'

\end{document}